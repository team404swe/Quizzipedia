\documentclass[a4paper,11pt]{article}
	%INCLUDE DEL TEMPLATE
	\input{../../template.tex}
	
	\title{\textbf{{\fontsize{8mm}{5mm}\selectfont QUIZZIPEDIA}}}
	
	\begin{document}
	\pagenumbering{Roman}
	\maketitle
	\thispagestyle{empty}

	\begin{center}
		\includegraphics{../../team_not_found.jpg}\\
		\vspace{20mm}
		\textbf{{\Large Glossario 1.0}}	
	\end{center}

	\thispagestyle{empty}	% per togliere il numero in fondo pagina
	\introtab{Glossario}					%1 nome documento
			{1.0} 							%2 versione
			{Esterno} 						%3 Uso
			{19 gennaio 2015} 				%4 Data cre
			{\today} 						%5 Data mod
			{Martin V. Mbouenda - D. Bortot}		%6 Redazione1
			{Andrea Multineddu} 			%8 Verifica
			{Davide Bortot}					%9 Approvazione
			
	\newpage
	\fancyfoot[R]{\thepage}
	\pagenumbering{Roman}
	\section*{Registro delle modifiche}
		\begin{longtable}{{|p{0.10\textwidth}|p{0.15\textwidth}|p{0.15\textwidth}|p{0.50\textwidth}|}} 
	 		\hline			
			\rigaregistro{\textbf{Versione}}{\textbf{Autore}}{\textbf{Data}}{\hspace{5 mm} \textbf{Descrizione}}
			\rigaregistro{1.0}{D. Bortot \newline (Responsabile)}{16/03/2016}{Approvazione del documento.}
			\rigaregistro{0.1}{A. Multineddu \newline (Verificatore)}{15/03/2016}{Verifica completa del documento.}
			\rigaregistro{0.0.7}{D. Bortot \newline (Responsabile)}{15/03/2016}{Aggiunta e definizione nuovi termini.}
			\rigaregistro{0.0.6}{D. Bortot \newline (Responsabile)}{12/03/2016}{Aggiunta e definizione nuovi termini.}
			\rigaregistro{0.0.5}{L. Alessio \newline (Verificatore)}{11/03/2016}{Aggiunta e definizione nuovi termini.}
			\rigaregistro{0.0.4}{D. Bortot \newline (Responsabile)}{10/03/2016}{Definiti i termini precedentemente aggiunti.}
			\rigaregistro{0.0.3}{M. Crivellaro \newline (Verificatore)}{09/03/2016}{Modifica struttura documento e aggiunta termini.}
			\rigaregistro{0.0.2}{D. Bortot \newline (Responsabile)}{08/03/2016}{Aggiunti termini.}			
			\rigaregistro{0.0.1}{M. Mbouenda \newline (Amministratore)}{19/01/2016}{Prima stesura del documento.}
			%\caption{Versionamento del documento} 
		\end{longtable}
		
\newpage
\tableofcontents
\pagenumbering{arabic}
\newpage
\fancyhead[R]{GLOSSARIO}
%\appendix  % cambia la numerazione delle sezioni da numeri a lettere
\sezioneglos{A}
\beginglos
	\itemglos{App}{abbreviativo di applicazione software; il termine identifica solitamente le applicazioni in ambito mobile, o specifici servizi web.}
	\itemglos{Astah}{software che fornisce gli strumenti necessari alla modellazione di sistemi. Nel caso informatico fornisce una piattaforma UML/ERD per il software development.}
\fineglos
%\newpage
%\sezioneglos{B}
%\beginglos
%\fineglos
\newpage
\sezioneglos{C}
\beginglos
\itemglos{CSS3}{l'ultima iterazione del linguaggio CSS (Cascading Style Sheets), usato per definire la formattazione di documenti HTML, XHTML e XML, ampiamente usato nel Web.}
\itemglos{Committente}{la figura che commissiona un lavoro, in questo caso specifico l'azienda Zucchetti S.P.A..}
\itemglos{Complessità \newline Ciclomatica}{una metrica software, sviluppata da Thomas J. McCabe nel 1976, utilizzata per misurare la complessità di un programma. Misura direttamente il numero di cammini linearmente indipendenti attraverso il grafo di controllo di flusso.}
\fineglos
\newpage
\sezioneglos{D}
\beginglos
\itemglos{Design Pattern}{nell'ambito dell'ingegneria del software, una soluzione progettuale generale, un modello logico da applicare alla risoluzione di un problema ricorrente.}
		\itemglos{Driver}{ un insieme di procedure, spesso scritte in assembly, che permette ad un sistema operativo di pilotare e interagire con un dispositivo hardware.}
\fineglos
%\newpage
%\sezioneglos{E}
%\beginglos
%\fineglos
\newpage
\sezioneglos{F}
\beginglos
\itemglos{Framework}{un'architettura logica di supporto, spesso un'implementazione logica di un particolare design pattern, su cui un software può essere progettato e realizzato, facilitandone lo sviluppo da parte del programmatore.}
\fineglos
\newpage
\sezioneglos{G}
\beginglos
\itemglos{Git}{un sistema software di controllo di versione distribuito}
		\itemglos{GitHub}{un servizio web di hosting per lo sviluppo di progetti software, che usa il sistema di controllo di versione Git.}
\fineglos
\newpage
\sezioneglos{H}
\beginglos
\itemglos{HTML5}{un linguaggio di markup per la strutturazione delle pagine web, pubblicato come W3C Recommendation da ottobre 2014.}
\fineglos
\newpage
\sezioneglos{I}
\beginglos
\itemglos{I/O}{con il termine I/O (input/output) si intendono tutte le interfacce messe a disposizione dal sistema operativo, o più in generale da qualunque sistema di basso livello, ai programmi per effettuare uno scambio di dati o segnali con altri programmi, col computer o con lo stesso sistema.}
\itemglos{Inspection}{tecnica di analisi statica il cui scopo è di controllare e correggere solo un numero limitato, predefinito in una lista di controllo, di tipologie di errori.}
\fineglos
\newpage
\sezioneglos{J}
\beginglos
\itemglos{Javascript}{un linguaggio di scripting orientato agli oggetti e agli eventi, comunemente utilizzato nella programmazione Web lato client.}
\fineglos
%\newpage
%\sezioneglos{L}
%\beginglos
%\fineglos
\newpage
\sezioneglos{M}
\beginglos
\itemglos{Maintainability Index}{un indice compreso tra 0 e 100 che rappresenta la qualità del codice sorgente in termine di mantenibilità.}
\itemglos{Markup}{un linguaggio di markup è un insieme di regole che descrivono i meccanismi di rappresentazione (strutturali, semantici o presentazionali) di un testo che, utilizzando convenzioni standardizzate, sono utilizzabili su più supporti.}
\itemglos{Metriche di Halstead}{sistema di metriche calcolabili staticamente sul codice che permettono di misurarne le proprietà e le relazioni che tra esse intercorrono.}
\itemglos{MySpell}{spell-checker integrato di default nell'editor TexMaker.}
\fineglos
\newpage
\sezioneglos{N}
\beginglos
\itemglos{Node.js}{un framework I/O event-driven relativo all'utilizzo server-side di Javascript.}
\fineglos
\newpage
\sezioneglos{O}
\beginglos
\itemglos{OpenDocument}{il formato OpenDocument (ODF), abbreviazione di "OASIS Open Document Format for Office Applications", è un formato aperto per file di documento per l'archiviazione e lo scambio di documenti per la produttività di ufficio, come documenti di testo, fogli di calcolo, diagrammi e presentazioni.}
\fineglos
\newpage
\sezioneglos{P}
\beginglos
\itemglos{Package}{un package è un meccanismo per organizzare classi in gruppi logici, ha lo scopo di riunire classi (o entità analoghe, quali interfacce ed enumerazioni) logicamente correlate.}
\itemglos{Plato}{uno strumento software liberamente reperibile sulla piattaforma GitHub, che permette la visualizzazione, analisi statica e analisi di complessità di codice Javascript.}
\itemglos{Proponente}{la persona fisica o giuridica che propone il contratto al fornitore.}
\fineglos
\newpage
\sezioneglos{Q}
\beginglos
\itemglos{QML}{acronimo di Quiz Markup Language, un linguaggio di markup adeguato a descrivere quiz. Il linguaggio può gestire risposte vero/falso, risposte a scelta multipla, testi e
immagini.}
\itemglos{Quesito}{sinonimo di quiz.}
\itemglos{Questionario}{insieme di quesiti.
		Viene sottoposto all'utente del sistema Quizzipedia.}
\itemglos{Quiz}{domanda proposta all'utente che richiede una risposta precisa.
	Può essere a risposta multipla, vero o falso, a completamento; 
	i quiz sono categorizzati in base all'argomento trattato.}
\fineglos
\newpage
\sezioneglos{R}
\beginglos
\itemglos{Repository}{contenitore pensato per aggregare tutti gli elementi utili al progetto in un unico luogo, in modo da facilitarne la modifica, il controllo degli accessi e il versionamento.}
\fineglos
\newpage
\sezioneglos{S}
\beginglos
\itemglos{Stub}{una porzione di codice utilizzata in sostituzione di altre funzionalità software. Uno stub può simulare il comportamento di codice esistente, ed essere temporaneo sostituto di codice ancora da sviluppare. Gli stub sono perciò molto utili durante il porting di software, l'elaborazione distribuita e in generale durante lo sviluppo di software e il software testing.}
\fineglos
\newpage
\sezioneglos{T}
\beginglos
\itemglos{Task}{un unità di lavoro, un compito assegnabile a persone specifiche, dotato di una data di inizio e di una deadline da rispettare.}
\itemglos{TexMaker}{è un editor LaTeX cross-platform e open source, sviluppato in Qt e rilasciato sotto licenza GPL.}
\itemglos{Tool}{un qualsiasi strumento di supporto all'attività d'interesse, in questo caso lo sviluppo del progetto.}
\itemglos{Trello}{è un applicazione web pensata per aiutare e semplificare la gestione di progetto.}
\fineglos
\newpage
\sezioneglos{U}

\beginglos
\itemglos{UML}{un linguaggio di modellazione e specifica basato sul paradigma orientato agli oggetti.}
\itemglos{UTF-8}{ è una codifica dei caratteri Unicode in sequenze di lunghezza variabile di byte.}
\fineglos
\newpage
\sezioneglos{V}

\beginglos
\itemglos{Validazione}{il processo teso a verificare che il prodotto ottenuto sia conforme ai requisiti utente.}
		\itemglos{Verifica}{processo che si occupa di accertare che l'esecuzione delle attività di processi svolti nella fase in esame non
abbia introdotto errori nel prodotto.}
\fineglos
\newpage
\sezioneglos{W}
\beginglos
\itemglos{W3C}{acronimo di World Wide Web Consortium; è un'organizzazione non governativa internazionale che ha come scopo quello di sviluppare tutte le potenzialità del World Wide Web. La principale attività svolta dal W3C consiste nello stabilire standard tecnici inerenti sia i linguaggi di markup che i protocolli di comunicazione del Web.}
\itemglos{Walkthrough}{tecnica di analisi statica che tramite una lettura critica a largo spettro del prodotto in esame punta a rivelarne i difetti, senza basarsi su presupposizioni.}
\fineglos

	\end{document}
	