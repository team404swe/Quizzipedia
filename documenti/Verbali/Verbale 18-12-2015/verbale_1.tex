\documentclass[a4paper,11pt]{article}
\input{../../template.tex}

\graphicspath{ {./Images/} }
\title{\textbf{{\fontsize{10mm}{6mm}\selectfont QUIZZIPEDIA}}}

\makeindex

\begin{document}
	\maketitle
	
	\begin{center}

	\includegraphics{../../team_not_found.jpg}\\	
	\fontsize{5mm}{3mm}\url{team404swe@gmail.com}\\
	\vspace{40mm}
	\textbf{ Verbale 1 }
	\end{center}
	\thispagestyle{empty}	% per togliere il numero in fondo pagina
	\introtab{Verbale 1}			%1 nome documento
			{1.0} 							%2 versione
			{Esterno} 						%3 Uso
			{8 gennaio 2016} 				%4 Data cre
			{\today} 						%5 Data mod
			{Davide Bortot}					%6 Redazione1
			%{} 							%7 Redazione2
			{Andrea Multineddu} 			%8 Verifica
			{Davide Bortot} 				%9 Approvazione
	
	\newpage
	\section{Sommario}
	Questo documento ha lo scopo di formalizzare l'incontro tra i membri del gruppo \textbf{Team404} avvenuto il 18/12/2015. Scopo dell'incontro era mettere le basi ed accordarsi sui metodi di comunicazione e sulla prima suddivisione dei ruoli tra gli elementi.
	\subsection{Generalità}
	\begin{itemize}
	\item\textbf{Data}: 18 Dicembre 2015.
	\item\textbf{Luogo}: Padova, Via Trieste 63, Dipartimento di Matematica Pura ed Applicata dell'Università di Padova, Aula 1C/150.
	\item\textbf{Ora d'inizio}: 13:00.
	\item\textbf{Durata}: 40 minuti.
	\item\textbf{Partecipanti}: Davide Bortot, Martin Vadice Mbouenda, Marco Crivellaro e Alex Beccaro.
	\end{itemize}

	\newpage
	\section{Argomenti discussi}
		\subsection{Nome del Gruppo}
		Ci sono state alcune proposte come "I giubilati", "I fuorilegge" e "Team404". Quest'ultima è la proposta che ha riscosso maggiore approvazione, quindi "Team404" è da considerarsi nome ufficiale del gruppo.
		\subsection{Mezzi di comunicazione e di gestione}
		Durante l'incontro sono stati proposti i seguenti canali di comunicazione e di collaborazione:
		\begin{itemize}
			\item \textbf{WhatsApp}: adatto per comunicazioni semplici e veloci, sarà il canale ufficiale di comunicazione interna.
			\item \textbf{Dropbox}: verrà creata una cartella condivisa su Dropbox in cui mettere ogni file utile relativo al progetto (documenti, codice, immagini, etc.).
			\item \textbf{E-mail}: creazione di un e-mail (Gmail) di riferimento del gruppo per le comunicazioni esterne ufficiali, principalmente gestita del Responsabile.
			\item \textbf{Skype}: mezzo comunicativo per decisioni importanti o complesse di gruppo in caso non sia possibile trovarsi personalmente. Valutabili anche alternative (TeamSpeak, altre).
			\item \textbf{Gestione di progetto}: sono state proposte tre diverse piattaforme web per la gestione di progetto: Trello, Asana, TeamWork. Serviranno ulteriori discussioni per decidere quale strumento utilizzare.
		\end{itemize}
		\subsection{Documentazione}
		Si è deciso che tutti i documenti dovranno essere creati in LaTeX, per fornire una struttura solida ma allo stesso tempo flessibile alle esigenze dei contenuti. Per assicurare coerenza di stile e formattazione verrà creato un file "template" da includere in ogni documento.
		\subsection{Ruoli}
		Si è deciso che durante il primo periodo del progetto i ruoli verranno così suddivisi:
		\begin{table}[h!]			
		\begin{center}
			\begin{tabular}{l c}
			\textbf{Componente} & \textbf{Ruolo}\\
			\midrule
			Davide Bortot & Responsabile\\
			Martin Vadice Mbouenda & Amministratore\\
			Marco Crivellaro & Analista\\
			Alex Beccaro & Analista\\
			Luca Alessio & Analista/Verificatore\\
			Andrea Multineddu & Analista/Verificatore\\
			\midrule
			\end{tabular}
		\end{center}
		\end{table}
		\newline
		Gli ultimi due membri citati dovranno accordarsi su quale ruolo ricoprire; preferibilmente la figura dell'analista dovrebbe essere disponibile a trovarsi fisicamente con gli altri due analisti.
\end{document}