	\usepackage[utf8]{inputenc}
	\usepackage[italian]{babel}
	\usepackage{hyperref}	%Consente l'inserimento di \url
	\usepackage{booktabs}	%Utilità di abbellimento tabelle
	\usepackage{longtable}
	\usepackage{tabularx}
	%\usepackage{widetable}
	\usepackage{array}
	\usepackage{listings}
	\usepackage{graphicx}
	\usepackage{fancyhdr}
	\usepackage[a4paper,top=3cm,bottom=3cm,left=2cm,right=2cm]{geometry}
	% *************************************
	% QUI CODICE PER \SUBSUBSUBSECTION
	\usepackage{titlesec}
	\titleclass{\subsubsubsection}{straight}[\subsection]
	
	\newcounter{subsubsubsection}[subsubsection]
	\renewcommand\thesubsubsubsection{\thesubsubsection.\arabic{subsubsubsection}}
	\renewcommand\theparagraph{\thesubsubsubsection.\arabic{paragraph}} % optional; useful if paragraphs are to be numbered
	
	\titleformat{\subsubsubsection}
	  {\normalfont\normalsize\bfseries}{\thesubsubsubsection}{1em}{}
	\titlespacing*{\subsubsubsection}
	{0pt}{3.25ex plus 1ex minus .2ex}{1.5ex plus .2ex}
	
	\makeatletter
	\renewcommand\paragraph{\@startsection{paragraph}{5}{\z@}%
	  {3.25ex \@plus1ex \@minus.2ex}%
	  {-1em}%
	  {\normalfont\normalsize\bfseries}}
	\renewcommand\subparagraph{\@startsection{subparagraph}{6}{\parindent}%
	  {3.25ex \@plus1ex \@minus .2ex}%
	  {-1em}%
	  {\normalfont\normalsize\bfseries}}
	\def\toclevel@subsubsubsection{4}
	\def\toclevel@paragraph{5}
	\def\toclevel@paragraph{6}
	\def\l@subsubsubsection{\@dottedtocline{4}{7em}{4em}}
	\def\l@paragraph{\@dottedtocline{5}{10em}{5em}}
	\def\l@subparagraph{\@dottedtocline{6}{14em}{6em}}
	\makeatother
	
	\setcounter{secnumdepth}{4}
	\setcounter{tocdepth}{4}
	%FINE \SUBSUBSUBSECTION
	%****************************************
	%STYLE PER INSERIMENTO DEL CODICE
	\lstdefinestyle{style1}{
	  belowcaptionskip=1\baselineskip,
	  breaklines=true,
	  frame=L,
	  xleftmargin=\parindent,
	  language=Pascal,
	  showstringspaces=false,
	  basicstyle=\footnotesize\ttfamily,
	  keywordstyle=\bfseries\color{blue},
	  commentstyle=\itshape\color{blue},
	  identifierstyle=\color{blue},
	  stringstyle=\color{orange},
	}
	
	\lstdefinestyle{style2}{
	  belowcaptionskip=1\baselineskip,
	  frame=L,
	  xleftmargin=\parindent,
	  language=C,
	  basicstyle=\footnotesize\ttfamily,
	  commentstyle=\itshape\color{blue},
	}
	\lstset{style=style1}
	
	%FINE STYLE INSERIMENTO CODICE
	%*****************************************
	\usepackage[default]{cantarell} %% Use option "defaultsans" to use cantarell as sans serif only
	\usepackage[T1]{fontenc}        %% for font
	\hypersetup{colorlinks, linkcolor=black, urlcolor=blue}
	\newcommand{\addglos}{\begin{tiny}{\textbf{\ped{G}}} \end{tiny}}
	\pagestyle{fancy}
	\fancyhead{}
	\fancyfoot{}
	%\fancyhead[L]{\includegraphics[scale=0.28]{team_not_found.jpeg}}
	\fancyhead[L]{\includegraphics[scale=0.15]{../team404_small.jpg} Quizzipedia}
	\fancyhead[R]{\leftmark}
	\fancyfoot[L]{Universit\`a degli studi di Padova - IS 2015/2016 \\ \url{team404swe@gmail.com}}

	
	%Commando usato per la tabella di informazioni sul documento
	\newcommand{\introtab}[9]{
		\begin{table}[ht]
		\begin{center}		
		\begin{tabular}{r l}			
			\toprule		
			\multicolumn{2}{c}{\textbf{ Informazioni sul documento }} \\
			\midrule 
			\textbf{Nome Documento}			& \vline \hspace{3.5 mm} {#1} \\
			\textbf{Versione}				& \vline \hspace{3.5 mm} {#2} \\
			\textbf{Uso} 					& \vline \hspace{3.5 mm} {#3} \\
			\textbf{Data Creazione} 		& \vline \hspace{3.5 mm} {#4} \\
			\textbf{Data Ultima Modifica} 	& \vline \hspace{3.5 mm} {#5} \\
			\textbf{Redazione}				& \vline \hspace{3.5 mm} {#6} \\
											%& \vline \hspace{3.5 mm} {#7} \\	
			\textbf{Verifica} 				& \vline \hspace{3.5 mm} {#7}	\\
			\textbf{Approvazione}			& \vline \hspace{3.5 mm} {#8}\\	
			\textbf{Committente} 			& \vline \hspace{3.5 mm} Zucchetti SPA\\
			\textbf{Lista di distribuzione} & \vline \hspace{3.5 mm} Prof. Vardanega Tullio \\
											& \vline \hspace{3.5 mm} Prof. Cardin Riccardo \\
											& \vline \hspace{3.5 mm} TEAM404 \\
	\bottomrule	
	\end{tabular}
	\end{center}
	\end{table}
	}
	% Comando di inizio del registro
	\newcommand{\beginregistro}{
		\begin{longtable}{{|p{0.10\textwidth}|p{0.15\textwidth}|p{0.15\textwidth}|p{0.50\textwidth}|}} 
	 		\hline	
	}
	% commando usato pr inserire una riga al registro delle modifiche
	\newcommand{\rigaregistro}[4]{
		#1 & #2 & #3 & #4 \\
			\hline	
	}
	% Comando di fine registro
	\newcommand{\fineregistro}{ \end{longtable}	}
	
	%************************************************
	% commandi per il GLOSSARIO
	%***********************************************
	% Commando di inizio tabella Glossario
	\newcommand{\beginglos}{
		\begin{longtable}{{p{0.20\textwidth}p{0.65\textwidth}}}	
	}
	% Commando per i termini del glossario
	
	\newcommand{\itemglos}[2]{
		\textbf{#1 :} & {#2} \\ \\ \\
	}
	% Commando fine Glossario
	\newcommand{\fineglos}{ \end{longtable} }